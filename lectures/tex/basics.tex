\section{Basics} % (fold)
\label{sec:basics}

\begin{frame}\frametitle{How to install Python}
    \framesubtitle{version 2.6+}

    Many alternatives, but I suggest installing using a prepackaged distribution, such as Anaconda

    \vfill

    \url{https://store.continuum.io/cshop/anaconda/}

    \vfill

    This is very easy to install and also comes with a lot of packages.

    \vfill

    See the getting started instructions on the course website for more information.

\end{frame}

\begin{frame}\frametitle{Packages}

    Packages enhance the capabilities of Python, so you don't have
    to progam everything by yourself (it's faster too!).

    \vfill

    For example: Numpy is a package that adds many linear algebra capabilities, more on that later

\end{frame}

\begin{frame}\frametitle{How to install packages}
    \framesubtitle{version 2.6+}

    To install a package that you do not have, use \texttt{pip},
    which is the Python package manager.

    \vfill

    such as

    \texttt{\$ pip install seaborn}

\end{frame}

\begin{frame}\frametitle{Python 3}

    Python 3 has been around for a while and is slowly gaining traction.

    However, many people still use Python 2, so we will stick with that.

    Differences are not too big, so you can easily switch.

\end{frame}

\begin{frame}\frametitle{Cloud9 demo}

    Please log into your Cloud9 account on c9.io

    \vfill

    Clone the code using:
    \texttt{\$ git clone https://github.com/schmit/hangman.git}

    \vfill

    If you don't have git and want a local copy, download the code using
    \url{https://github.com/schmit/hangman/archive/master.zip}

\end{frame}

\begin{frame}\frametitle{How to use Python}
    \framesubtitle{}

    There are two ways to use Python:

    \vfill

    command-line mode: talk directly to the interpreter

    \vfill

    scripting-mode: write code in a file (called script) and run code by typing\\
    \begin{center}
        \texttt{\$ python scriptname.py}
    \end{center}
    in the terminal.
    \vfill
    The latter is what we will focus on in this course, though using the command-line
    can be useful to quickly check functionality.
\end{frame}


\begin{frame}\frametitle{The interpreter}
    \framesubtitle{Using Python as your calculator}

    We can start the intepreter by typing `python' in the terminal.

    \vfill

    Now we can interactively give instructions to the computer, using the Python language.

    \vfill

    \begin{figure}
        \centering
        \centering
        \includegraphics[height=2.0in]{"img/python_terminal"}
        \label{fig:terminal}
    \end{figure}

\end{frame}

\begin{frame}\frametitle{Scripting mode}
    \framesubtitle{}

    A more conveniened way to interact with Python is to write a script.

    \vfill

    A script contains all code you want to execute. Then you call Python on
    the script to run the script.

    \vfill

    First browse, using the terminal, to where the script is saved

    \vfill

    Then call \texttt{python scriptname.py}

\end{frame}

\begin{frame}[fragile]\frametitle{Scripting mode}
    \framesubtitle{An example}

    Suppose the Python script is saved in a folder \texttt{~/Documents/Python}
    called \texttt{firstscript.py}.

    \vfill

    Then browse to the folder by entering the following command into the terminal

    \vfill

    \verb|$ cd ~/Documents/Python|

    \vfill

    And then run the script by entering

    \vfill

    \verb|$ python firstscript.py|

\end{frame}

\begin{frame}\frametitle{Print statement}
    \framesubtitle{}

    We can print output to screen using the \texttt{print} command

    \codeblock{code/basics_hello.py}

\end{frame}

% section basics (end)

\section{Variables} % (fold)
\label{sec:variables}

\begin{frame}\frametitle{Values}
    \framesubtitle{}

    A value is the fundamental thing that a program manipulates.

    \vfill

    Values can be ``Hello, world!'', \texttt{42}, \texttt{12.34}, \texttt{True}

    \vfill

    Values have types\ldots

\end{frame}

\begin{frame}\frametitle{Types}
    \framesubtitle{}

\begin{description}
    \item[\textbf{Bool}ean] True/False
    \item[\textbf{Str}ing] ``Hello, world!''
    \item[\textbf{Int}eger] 92
    \item[\textbf{Float}] 3.1415
\end{description}

\vfill

Use \texttt{type} to find out the type of a variable, as in

\texttt{>>> type("Hello, world!")}\\
\texttt{<type `str'>}

\end{frame}

\begin{frame}\frametitle{Variables}
    \framesubtitle{}

    One of the most basic and powerful concepts is that of a \emph{variable}.

    \vfill

    A variable \emph{assigns} a name to a value.

    \codeblock{code/basics_var.py}

    Try it!

\end{frame}

\begin{frame}\frametitle{Variables}
    \framesubtitle{}

    Almost always preferred to use variables over values:

    \begin{itemize}
        \item Easier to update code
        \item Easier to understand code (useful naming)
    \end{itemize}

    \vfill
    What does the following code do:\\
    \texttt{print 4.2 * 3.5}

    \pause

    \codeblock{code/basics_var2.py}

\end{frame}

% \begin{frame}\frametitle{Naming}
%     \framesubtitle{}

%     \begin{itemize}
%         \item Choose meaningful variable names
%         \item Should start with a letter
%         \item Cannot use keywords
%     \end{itemize}

%     \vfill

%     Use the Google Python style guide!\footnote{See References on course website.}

% \end{frame}


\begin{frame}\frametitle{Keywords}
    \framesubtitle{}

    Not allowed to use keywords, they define structure and rules of a language.

    Python has 29 keywords, they include:

    \begin{itemize}
        \item and
        \item def
        \item for
        \item return
        \item is
        \item in
        \item class
    \end{itemize}


\end{frame}

% \begin{frame}\frametitle{Statements, expressions and operators}
%     \framesubtitle{}

%     A statement is an instruction that Python can execute, such as

%     \texttt{x = 3}

%     \vfill

%     \emph{Operators} are special symbols that represent computations, like addition,
%     the values they \emph{operate} on are called operands

%     \vfill

%     An \emph{expression} is a combination of values, variable and operators

%     \texttt{x + 3}

% \end{frame}

\begin{frame}\frametitle{Integers}
    \framesubtitle{1, 2, -32, 45}

    Operators for integers

    \texttt{+ - * / \% **}

    \vfill

    Note: / uses integer division:

    \texttt{5 / 2} yields \texttt{2}

    \vfill

    But, if one of the operands is a float, the return value is a float:

    \texttt{5 / 2.0} yields \texttt{2.5}

    \vfill

    Note: Python automatically uses long integers for very large integers.

\end{frame}

\begin{frame}\frametitle{Floats}
    \framesubtitle{1.23, 6.43, 42.0}

    A floating point number approximates a real number.

    Note: only finite precision, and finite range (overflow)!

    Operators for floats

    \begin{description}
        \item[+] addition
        \item[-] subtraction
        \item[*] multiplication
        \item[/] division
        \item[**] power
    \end{description}

\end{frame}


\begin{frame}\frametitle{Booleans}
    \framesubtitle{True, False}

    \textbf{Boolean expressions}:

    \begin{description}
        \item[$==$] equals: \texttt{5 == 5} yields \texttt{True}
        \item[$!=$] does not equal: \texttt{5 != 5} yields \texttt{False}
        \item[$>$] greater than: \texttt{5 > 4} yields \texttt{True}
        \item[$>=$] greater than or equal: \texttt{5 >= 5} yields \texttt{True}
    \end{description}

    Similarly, we have $<$ and $<=$.

    \pause\vfill

    \textbf{Logical operators}:
    \begin{description}
        \item \texttt{True and False} yields \texttt{False}
        \item \texttt{True or False} yields \texttt{True}
        \item \texttt{not True} yields \texttt{False}
    \end{description}

\end{frame}

\begin{frame}\frametitle{Modules}
    \framesubtitle{}

    Not all functionality avaible comes automatically when starting python, and
    with good reasons.

    \vfill

    We can add extra functionality by importing modules:

    \vfill

    \texttt{>>> import math}\\
    \texttt{>>> math.pi}\\
    \texttt{3.141592653589793}

    \vfill

    Useful modules: \texttt{math}, \texttt{string}, \texttt{random}, and as we will see later
    \texttt{numpy}, \texttt{scipy} and \texttt{matplotlib}.

    \vfill

    More on modules later!

\end{frame}

% \begin{frame}\frametitle{Final words}
%     \framesubtitle{}

%     No need to remember the details: everything is well documented online!

%     \vfill

%     However, knowing about the tools that exist will help you look them up!

%     \vfill

%     Always try the Python documentation first if you forgot something.


% \end{frame}

% section variables (end)

% \section{What is a variable?}
% \begin{frame}
% \frametitle{Basic variables}

% A variable stores information

% \vspace{0.2in}

% In Python this is simple:

% \codeblock{code/basic_vars1.py}

% \textcolor{comment-color}{\#} signifies the start of a comment in Python.  The comment terminates at the end of the line.

% \end{frame}


% \begin{frame}
% \frametitle{Basic variables}

% Variables can change (they are \emph{variable})

% \codeblock{code/basic_vars2.py}


% (no more knowledge of 1.343, 2, or 14)

% \end{frame}

% \begin{frame}
% \frametitle{Types}
% Unlike C/C++ and Java, variables can change types. Python keeps track of the type internally.

% \codeblock{code/basic_vars3.py}

% If you have PL theory background: Python is strongly typed but not statically typed

% \end{frame}


% \section{Arithmetic and boolean operators}

% \begin{frame}
% \frametitle{Arithmetic operators}

% \codeblock{code/arith_ops1.py}


% Python supports this arithmetic on strings.  (Note: Matlab does not.)
% \end{frame}

% \begin{frame}
% \frametitle{Arithmetic operators}

% Shorthand to combine assignment and addition statements:

% \codeblock{code/arith_ops2.py}


% \vspace{0.1in}

% \texttt{x} is now \texttt{6} and \texttt{y} is now \textcolor{string-color}{\texttt{'hihihihi'}}

% \end{frame}


% \begin{frame}
% \frametitle{Comparison operators}

% \codeblock{code/comp_ops1.py}


% \end{frame}

% \begin{frame}
% \frametitle{Boolean operations}

% \codeblock{code/bool_ops1.py}


% Python conveniently uses the keywords \textcolor{teal}{and}, \textcolor{teal}{or}, \textcolor{teal}{not} instead of the symbols \textcolor{teal}{\&\&}, $\textcolor{teal}{||}$, \textcolor{teal}{!}

% \end{frame}

% \section{Control Flow}
% \begin{frame}
% \frametitle{if}
% The \texttt{if} statement is the most basic way to control the direction and flow of a program

% \codeblock{code/if1.py}


% \end{frame}


% \begin{frame}
% \frametitle{if/else}
% \texttt{if} is often accompanied by \texttt{else} to specify a second path for the code if the \texttt{if} statement is false

% \codeblock{code/if2.py}


% \end{frame}

% \begin{frame}
% \frametitle{Evaluating numerical values}

% As boolean values, the numerical value 0 is False and all other numerical values are True (1, 4.33, -12, ...).

% \codeblock{code/bool_ops2.py}


% \end{frame}

% \begin{frame}
% \frametitle{while}
% The \texttt{while} loop repeatedly executes a task while a condition is true

% \codeblock{code/while1.py}

% \end{frame}

% \begin{frame}
% \frametitle{while}

% \codeblock{code/while2.py}

% What is this code doing?


% \end{frame}

% \begin{frame}
% \frametitle{for}
% The \texttt{for} loop also repeatedly executes a task

% \vspace{0.1in}

% Typically, \texttt{for} provides more structure:

% \vspace{0.1in}

% \centering
% for ($i = 0$; $i < n$; $i = i + 1$)

% \end{frame}

% %\begin{frame}
% %\frametitle{for}

% %\begin{figure}[h!]
% %\centering
% %\includegraphics[height=1.5in]{"images/for"}
% %\caption{}
% %\label{}
% %\end{figure}
% %\end{frame}

% \begin{frame}
% \frametitle{Python for}
% \begin{itemize}
% \setlength{\itemsep}{0.2in}
% \item{Python uses \texttt{for} differently than Matlab, C++, ...}
% \item{\texttt{for} is used to iterate over elements in an ``object"}
% \item{This is one reason why Python is easy and powerful}
% \item{More next lecture on how you can iterate over a ``object"}
% \end{itemize}

% \end{frame}

% \begin{frame}
% \frametitle{Python for}

% \codeblock{code/for1.py}

% \vspace{0.5in}

% more next lecture...

% \end{frame}
