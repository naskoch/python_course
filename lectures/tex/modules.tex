\section{Modules} % (fold)
\label{sec:modules}

\begin{frame}[fragile]\frametitle{Importing a module}
    \framesubtitle{}

    We can import a module by using \texttt{import}

    \vfill

    E.g. \texttt{import math}

    \vfill

    We can then access everything in \texttt{math}, for example the square root
    function, by:

    \vfill

    \texttt{math.sqrt(2)}


\end{frame}

\begin{frame}[fragile]\frametitle{Importing as}
    \framesubtitle{}

    We can rename imported modules

    \vfill

    E.g. \texttt{import math as m}

    \vfill

    Now we can write \texttt{m.sqrt(2)}

\end{frame}

\begin{frame}\frametitle{In case we only need some part of a module}
    \framesubtitle{}

    We can import only what we need using the \texttt{from ... import ...} syntax.

    \vfill

    E.g. \texttt{from math import sqrt}

    \vfill

    Now we can use \texttt{sqrt(2)} directly.

\end{frame}

\begin{frame}\frametitle{Import all from module}
    \framesubtitle{}

    To import all functions, we can use \texttt{*}:

    \vfill

    E.g. \texttt{from math import *}

    \vfill

    Again, we can use \texttt{sqrt(2)} directly.

    \vfill

    Note that this is considered bad practice!
    It makes it hard to understand where functions come from and what if several modules come with functions with same name.

\end{frame}

\begin{frame}\frametitle{Writing your own modules}
    \framesubtitle{}

    It is perfectly fine to write and use your own modules. Simply
    import the name of the file you want to use as module.

    \vfill

    E.g.

    \codeblock{code/modules_first.py}

    \texttt{import firstmodule}\\
    \texttt{firstmodule.helloworld()}

    \vfill

    What do you notice?

\end{frame}

\begin{frame}\frametitle{Only running code when main file}
    \framesubtitle{}

    By default, Python executes all code in a module when we import it.
    However, we can make code run only when the file is the main file:

    \codeblock{code/modules_second.py}

    Try it!

\end{frame}

% section modules (end)
