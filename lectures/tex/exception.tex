\section{Exception handling} % (fold)
\label{sec:exception_handling}


\begin{frame}\frametitle{Wordcount}
    \framesubtitle{An example}

    Recall the wordcount exercise:

    \vfill

    Write a function that takes a filename, and returns the 20 most common words.

\end{frame}

\begin{frame}\frametitle{}
    \framesubtitle{}

    Suppose we have written a function \texttt{topkwords(filename, k)}

    \vfill

    Instead of entering filename and value of k in the script, we can also run it
    from the terminal.

\end{frame}

\begin{frame}\frametitle{Parse input from command line}
    \framesubtitle{}

    The \texttt{sys} module allows us to read the terminal command that
    started the script:

    \codeblock{code/exception_argv.py}

    \pause\vfill

    \texttt{sys.argv} holds a list with command line arguments passed to a
    Python script.

\end{frame}

\begin{frame}\frametitle{Back to the wordcount example}
    \framesubtitle{}

    \codeblock{code/exception_wc.py}

    \pause

    Issues?

\end{frame}

\begin{frame}\frametitle{Issues}
    \framesubtitle{}

    \begin{itemize}
        \item What if the file does not exist?
        \item What if the second argument is not an integer?
        \pause
        \item What if no command line arguments are supplied?
    \end{itemize}
    \pause

    All result in errors:

    \begin{itemize}
        \item IOError
        \item ValueError
        \item IndexError
    \end{itemize}

\end{frame}

\begin{frame}\frametitle{Exception handling}
    \framesubtitle{}

    What do we want to happen when these errors occur? Should the program simply crash?

    \pause

    No, we want it to gracefully handle these

    \begin{itemize}
        \item IOError: Tell the user the file does not exist.
        \item ValueError, IndexError: Tell the user what the format of the command
        line arguments should be.
    \end{itemize}

\end{frame}

\begin{frame}\frametitle{Try ... Except}
    \framesubtitle{}

    \codeblock{code/exception_wc2.py}

\end{frame}

\begin{frame}\frametitle{Try ... Except}

    \begin{itemize}
        \item The try clause is executed
        \item If no exception occurs, the except clause is skipped
        \pause
        \item If an exception occurs, the rest of the try clause is skipped.
        Then if the exception type is matched, the except clause is executed.
        Then the code continues after the try statement
        \pause
        \item If an exception occurs with no match in the except clause, execution
        is stopped and we get the standard error
    \end{itemize}

\end{frame}

\begin{frame}\frametitle{A naked except}
    \framesubtitle{}

    We can have a naked except that catches any error:
    \codeblock{code/exception_except.py}

    Use this with extreme caution though, as genuine bugs might be impossible
    to correct!

\end{frame}

\begin{frame}\frametitle{Try Except Else}
    \framesubtitle{}

    Else clause after all except statements is executed after successful
    execution of the try block (hence, when no exception was raised)

    \codeblock{code/exception_else.py}

    \textbf{Why?} Avoids catching exception that was not protected

    E.g. consider \texttt{f.readlines} raising an \texttt{IOError}

    \vfill

    \scriptsize{Example from Python docs}

\end{frame}

\begin{frame}[fragile]\frametitle{Raise}
    \framesubtitle{}

    We can use \texttt{Raise} to raise an exception ourselves.

    \begin{verbatim}
>>> raise NameError('Oops')
Traceback (most recent call last):
  File "<stdin>", line 1, in ?
NameError: Oops
    \end{verbatim}

\end{frame}

% \begin{frame}[fragile]\frametitle{Raise (2)}
%     \framesubtitle{}

%     Furthermore, we can use raise to re-raise an exception that we
%     do not intend to handle

%     \begin{verbatim}
% >>> try:
% ...     raise NameError('Oops')
% ... except NameError:
% ...     print 'An exception flew by!'
% ...     raise
% ...
% An exception flew by!
% Traceback (most recent call last):
%   File "<stdin>", line 2, in ?
% NameError: Oops
%     \end{verbatim}

% \end{frame}

\begin{frame}\frametitle{Finally}
    \framesubtitle{}

    The \texttt{finally} statement is always executed before leaving the
    \texttt{try} statement, whether or not an exception has occured.

    \codeblock{code/exception_finally.py}

    \pause

    Useful in case we have to close files, closing network connections etc.

\end{frame}

\begin{frame}\frametitle{Raising our own exceptions: Rational class}

    Recall the Rational class we considered a few lectures ago:

    \codeblock{code/classes_rat_init.py}

    What if \texttt{q = 0}?

\end{frame}

\begin{frame}\frametitle{Rational class}

\codeblock{code/exception_rational.py}

... Not really anything!

\vfill
\scriptsize{Rational class: \url{https://gist.github.com/schmit/875840d1a231526b572e}}

\end{frame}

\begin{frame}\frametitle{Making the necessary change}

If \texttt{q = 0}, then raise an exception!

Which one?

\pause

\codeblock{code/exception_rational_fix.py}

In the above example, where is the exception raised?

\end{frame}

% section exception_handling (end)
