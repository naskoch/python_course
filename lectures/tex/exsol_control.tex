\section{Exercise solutions: control flow} % (fold)
\label{sec:exercise_solutions_control_flow}

\begin{frame}\frametitle{For loops}
    \framesubtitle{Problem}

    Print the numbers 1 to 100 that are divisible by 5 but not by 3

\end{frame}

\begin{frame}\frametitle{For loops}
    \framesubtitle{Solution}

    \codeblock{code/exsol_control_for.py}

\end{frame}

\begin{frame}\frametitle{While loops}
    \framesubtitle{Exercise}

    The smallest number that is divisible by 2, 3 and 4 is 12.
    Find the smallest number that is divisible by all integers between 1 and 10.

\end{frame}

\begin{frame}\frametitle{While loop}
    \framesubtitle{Solution}

    \codeblock{code/exsol_control_while.py}

\end{frame}


\begin{frame}\frametitle{Collatz sequence}
    \framesubtitle{Problem}

    A Collatz sequence is formed as follows:
    We start with some number $x_0$, and we find the next number in the sequence by
    \[
        x_{i+1} = \begin{cases}
            x_i / 2 & \text{ if $x_i$ is even}\\
            3x_i + 1 & \text{ if $x_i$ is odd}
        \end{cases}
    \]
    If $x_i = 1$, we stop iterating and have found the full sequence.

    It is conjectured, though not proven, that every chain eventually ends at $1$.

    Print the Collatz sequence starting at $x_0 = 103$.

\end{frame}

\begin{frame}\frametitle{Collatz sequence}
    \framesubtitle{Solution}

    \codeblock{code/exsol_control_collatz.py}

\end{frame}

% section exercise_solutions_control_flow (end)
