\section{Lists} % (fold)
\label{sec:lists}

% This sheet is a little long

\begin{questions}



\titledquestion{Short questions} % (fold)
\label{sub:short_questions}

\begin{parts}
    \part Write a function that prints the elements of a list
    \part Write a function that prints the elements of a list in reverse
    \part Write your own implementation of the \verb|len| function
    that returns the number of elements in a list.
\end{parts}
% titledquestion short_questions (end)

\titledquestion{Copying lists} % (fold)
\label{sub:copying_lists}

\begin{parts}
    \part Create a list \verb|a| with some entries.
    \part Now set \verb|b = a|
    \part Change \verb|b[1]|
    \part What happened to \verb|a|?
    \part Now set \verb|c = a[:]|
    \part Change \verb|c[2]|
    \part What happened to \verb|a|?
\end{parts}

Now create a function \verb|set_first_elem_to_zero(l)| that takes a list, sets its
first entry to zero, and returns the list.

What happens to the original list?

\titledquestion{Lists of lists}

What is the difference between \texttt{a} and \texttt{b}:

\verb|a = [[]] * 3|

\verb|b = [[] for _ in xrange(3)]|

% titledquestion copying_lists (end)

\titledquestion{Lists and functions} % (fold)
\label{sub:lists_and_functions}

Write a function that takes a list and an index, and sets the value of the list
at the given index to 0.

% titledquestion lists_and_functions (end)

\titledquestion{Primes} % (fold)
\label{sub:primes}

In Section~\ref{sec:functions} you wrote a function that prints all primes up to $n$, and
a function that prints the first $n$ primes.
Update these functions such that they return lists instead.

% titledquestion primes (end)

\titledquestion{List comprehensions} % (fold)
\label{sub:list_comprehensions}

Let $i, j = 1,\ldots, n$

\begin{parts}
    \part Generate a list with elements \verb|[i,j]|.
    \part Generate a list with elements \verb|[i,j]| with $i < j$
    \part Generate a list with elements \verb|i + j| with both $i$ and $j$ prime and $i > j$.
    \part Write a function that evaluates an arbitrary polynomial $a_0 + a_1 x + a_2 x^2 + \ldots + a_n x^n$
    using a list comprehension, where you are given \verb|x| and a list with coefficients \verb|coefs|
    (hint: use enumerate)
\end{parts}

% titledquestion list_comprehensions (end)

\titledquestion{Filter} % (fold)
\label{sub:filter}

In lecture we have seen how to implement \texttt{map} using list comprehensions.
Implement \texttt{filter} using list comprehensions.
Name your functions \texttt{myfilter} so you can compare with Python's standard filter.
% titledquestion filter (end)

\titledquestion{Flatten a list of lists} % (fold)
\label{sub:flatten_a_list_of_lists}

Consider having a list with lists as elements, e.g.
\texttt{[[1,3], [3,6]]}.

Write a function that takes such a list, and returns a list with as elements the
elements of the sublists, e.g. \texttt{[1, 3, 3, 6]}.

% titledquestion flatten_a_list_of_lists (end)

\titledquestion{Finding the longest word} % (fold)
\label{sub:finding_longest_word}

Write a function that returns the longest word in a variable \texttt{text}
that contains a sentence.
While \texttt{text} may contain punctuation, these should not be taken into account.
What happens with ties?

As an example, consider: ``Hello, how was the football match earlier today???''

% titledquestion finding_longest_word (end)

\titledquestion{Collatz sequence, part 2} % (fold)
\label{sub:collatz_sequence_part_2}

Recall the Collatz sequence problem from Section~\ref{sub:collatz_sequence}.
Our goal is to find the number $n < 1,000,000$ that leads to the longest Collatz sequence.

\begin{parts}
    \part Write a function that for any $n$, returns its Collatz sequence as a list
    \part Write a function that finds the integer $x$ that leads to the longest
    Collatz sequence with $x < n$.
\end{parts}

% titledquestion collatz_sequence_part_2 (end)

\titledquestion{Pivots}
\label{sub:sorting}

Write a function that takes a value \texttt{x} and a list \texttt{ys},
and returns a list that contains the value \texttt{x} and all elements of \texttt{ys}
such that all values \texttt{y} in \texttt{ys} that are smaller than \texttt{x}
come first, then we element \texttt{x} and then the rest of the values in \texttt{ys}

For example, the output of
\texttt{f(3, [6, 4, 1, 7])} should be \texttt{[1, 3, 6, 4, 7]}

% titledquestion sorting (end)

\titledquestion{Prime challenge}
\label{sub:prime_challenge}

Write the function \texttt{primes(n)} that return a list with all prime numbers
up to \texttt{n} using three (or less) lines of code.

Hint 1: Use lambda functions and list comprehensions.

Hint 2: Use the first two lines to define two helper (lambda) functions.

% factors = lambda n: [x for x in xrange(1, n+1) if n % x == 0]
% prime = lambda n: factors(n) == [1, n]
% primes = lambda n: [x for x in xrange(2, n+1) if prime(x)]



\end{questions}

% section lists (end)
