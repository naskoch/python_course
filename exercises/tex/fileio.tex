\section{File I/O} % (fold)
\label{sec:file_i_o}
\begin{questions}

\titledquestion{Open a file} % (fold)
\label{sub:open}

    Write a function that opens a file (input: filename), and prints the
    file line by line.

% titledquestion open (end)

\titledquestion{Wordcount} % (fold)
\label{sub:wordcount}

    On the course website you can find a text file containing the complete works
    of William Shapespeare.

    \begin{parts}
        \item Find the 20 most common words
        \item How many unique words are used?
        \item How many words are used at least 5 times?
        \item Write the 200 most common words, and their counts, to a file.
    \end{parts}

% titledquestion wordcount (end)

\titledquestion{Random text generator II}
\label{sub:random_text_generator_ii}

In this exercise, we will finish the implementation of our random text generator
from Question~\ref{sec:dictionaries}.\ref{sub:random_text_generator}.

Change the \texttt{process\_textfile} function such that it reads a file line by
line and processes it, instead of using the hardcoded sample.
This should require only a few changes to the code.

Generate some random sentences based on the books provided in the \texttt{data/}
folder.

As you will notice, most phrases are very random.
To make the phrases seem a little more realistic, we should use not just
the last word, but the two last words.
Going back to our previous example: `the fire and the wind.'
we would have the (state, new\_word) pairs: (`the fire', `and'), (`fire and', `the'),
(`and the', `wind') etc.
Update your code to use states consisting of two words instead of one.

Bonus: Using more than 2 words is infeasible unless we have massive amounts of text
data because the number of states gets increasingly large.
Change your code such that you can specify \texttt{k}, the number of words per state.

\titledquestion{Sum of lists} % (fold)
\label{sub:sum_of_lists}

    \emph{Before you start coding, please read the entire problem.}
\begin{parts}
    \part{Data generation}

    Write a function that takes three integers, $n$, $a$ and $b$ and a filename
    and writes to the file a list with $n$ random integers between $a$ and $b$.

    \part{Reading the data}

    Write a function that can read the files as generated above and return the values.

    \part{Sum problem}

    Write a function that given two filenames (pointing to files as generated by the above function) and an integer $k$, finds all $u$ and $v$ such that $u + v = k$, and $u$ is an element of the first list and $v$ is a member of the second list.

    \part{Testing}

    Test your functions by generating 2 files with $n = 2000$, $a = 1$, $b = 10000$
    and $k = 5000$ and $k = 12000$.

    \part{Bonus: Efficiency}

    If you are up to a challenge, write a function that solves the sum problem with
    the restriction that you can only go over every number in the both lists once.

\end{parts}
% titledquestion sum_of_lists (end)

\end{questions}
% section file_i_o (end)
