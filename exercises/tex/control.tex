\section{Control flow} % (fold)
\label{sec:control_flow}
Disclaimer: Some of the following problems are inspired by problems from \url{www.projecteuler.net}.
Have a look if you are interested, there are some great challenges and Python is an excellent tool for solving them.


\begin{questions}
\titledquestion{Range}
    Type \texttt{range(5)} in the interpreter, what does the interpreter return?
    So what does \texttt{for i in range(5)} mean?

    Let's also find out whether the interpreter can help us understand the object `\texttt{range(5)}' better.
    Type \texttt{type(range(5))} in the interpreter.
    More on this soon!

\titledquestion{For loops}
    Use a \texttt{for} loop to:

    \begin{parts}
        \part Print the numbers 0 to 100
        \part Print the numbers 0 to 100 that are divisible by 7
        \part Print the numbers 1 to 100 that are divisible by 5 but not by 3
        \part Print for each of the numbers $x = 2,\ldots 20$,
            all numbers that divide $x$, excluding $1$ and $x$.
            Hence, for \texttt{18}, it should print \texttt{2 3 6 9}.
    \end{parts}

    Hint: see \url{https://docs.python.org/2.7/library/functions.html#range}.

\titledquestion{Simple while loops} % (fold)
    Instead of using a for loop, use a while loop to:

    \begin{parts}
        \part Print the numbers 0 to 100
        \part Print the numbers 0 to 100 that are divisible by 7
    \end{parts}

\titledquestion{Hangman update 1}
\label{sub:hangman_1}

    Let's reconsider the hangman code we saw in class.\footnote{To obtain the Hangman code, either use \texttt{\$ git clone https://github.com/schmit/hangman.git} (if you have git / use Cloud9)
    or dowload the code directly: \url{https://github.com/schmit/hangman/archive/master.zip}}
    We noted that the computer agent is not very good at guessing.
    Update the code such that the computer guesses `e' first, and 'a' second.

    Use the simulate.py script to see if this improves performance.

    Feel free to play around and see if you can do better!

\titledquestion{While loops} % (fold)
\label{sub:while_loop}

    Use a \texttt{while} loop to find the first 20 numbers that are divisible by 5, 7 and 11, and print them
    Hint: store the number found so far in a variable.

    Pseudo-code:

    \begin{verbatim}
        number found = 0
        x = 11
        while number found is less than 20:
            if x is divisible by 5, 7 and 11:
                print x
                increase number found by 1
            increase x by 1
    \end{verbatim}
% titledquestion while_loop (end)

\titledquestion{More while loops} % (fold)
\label{sub:more_while_loops}

    The smallest number that is divisible by 2, 3 and 4 is 12.
    Find the smallest number that is divisible by all integers between 1 and 10.

% titledquestion more_while_loops (end)

\titledquestion{Collatz sequence} % (fold)
\label{sub:collatz_sequence}

A Collatz sequence is formed as follows:
We start with some number $x_0$, and we find the next number in the sequence by
\[
    x_{i+1} = \begin{cases}
        x_i / 2 & \text{ if $x_i$ is even}\\
        3x_i + 1 & \text{ if $x_i$ is odd}
    \end{cases}
\]
If $x_i = 1$, we stop iterating and have found the full sequence.

For example, if we start with $x_0 = 5$, we obtain the sequence:
\begin{verbatim}
    5 16 8 4 2 1
\end{verbatim}

It is conjectured, though not proven, that every chain eventually ends at $1$.

Print the Collatz sequence starting at $x_0 = 103$.

% titledquestion collatz_sequence (end)
\end{questions}

% section control_flow (end)



