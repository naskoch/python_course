\section{Iterators} % (fold)
\label{sec:iterators}
\begin{questions}
\titledquestion{Collatz sequence} % (fold)
\label{sub:collatz_sequence}

Write a generator that generates the Collatz sequence with initial value $n$.
Use this to print out the Collatz sequence started at 103

Recall the Collatz sequence problem from last week.
A Collatz sequence is formed as follows:
We start with some number $x_0$, and we find the next number in the sequence by
\[
    x_{i+1} = \begin{cases}
        x_i / 2 & \text{ if $x_i$ is even}\\
        3x_i + 1 & \text{ if $x_i$ is odd}
    \end{cases}
\]
If $x_i = 1$, we stop iterating and have found the full sequence.

% titledquestion collatz_sequence (end)

\titledquestion{Collatz array using Numpy} % (fold)
\label{sub:collatz_array}

Use the Collatz generator you wrote in the previous exercise to generate a vector
with as ellements the Collatz sequence started at 61.

Hint: use the \texttt{np.fromiter} function.

% titledquestion collatz_array (end)

\titledquestion{Prime numbers} % (fold)
\label{sub:prime_numbers}

Write an iterator that iterates over the first $n$ prime numbers.
Use this to print out the first 10,000 primes.

% titledquestion prime_numbers (end)

\end{questions}
% section iterators (end)
